\section*{Konferenzbeiträge}
\begin{enumerate}[label=,leftmargin=0cm,itemsep=10pt]
\item „Vert.x and Scala - a reactive love affair“ / ScalaDays 2017, Reactive Meetup Hamburg, JUG Nuremberg
\item „Reactive DDD“ / JavaLand 2017
\item „Spark Training“ / ScalaDays 2016
\item „Reactive Microservices mit Vert.x 3“ / W-JAX 2015, Microservice Meetup Berlin, JUG Darmstadt, JavaLand 2016
\item „Event Sourcing FTW“ / JavaLand 2015, Coding Serbia 2015, Bed-Con 2015, Ludwig-Maximilians-Universität München
\item „Continuous Load Testing with Gatling“ / JUG Frankfurt
\item „Reactive Streams“ / parallelCon 2015
\item „Vert.x for World Domination" / W-JAX 2014, BedCon 2014
\item „Wicket 6" / BedCon 2013, JUG Ostfalen
\item „Hibernate Performance Tuning" / BedCon 2013
\item „MongoDB" / Senacor DevCon 2013
\item „Hibernate Performance Tuning" / BedCon 2012
\item „Wicket + JEE6 Fullstack" / BedCon 2012
\end{enumerate}

\section*{Fachartikel}
\begin{enumerate}[label=,leftmargin=0cm,itemsep=10pt]
\item Mader, Jochen: Modulare Software: Java 9-Primer (not yet released), in: iX (2017)
\item Mader, Jochen: Big Data, Fast Data: Shortcuts 195, in: Buch bei Entwickler Press (2016)
\item Mader, Jochen: Die Daten müssen irgendwie ins System: Daten mit Kafka und Akka annehmen, in: JavaMagazin (2016) 
\item Mader, Jochen: Spark, Mesos, Akka, Cassandra, Kafka: Aus Big Data werde Fast Data, in: JavaMagazin (2016)
\item Mader, Jochen: ETCD 2: Verteile und beherrsche!, in: Heise Developer (2015)
\item Mader, Jochen: Vert.x 3: Sieg in Runde 3, in: Java Magazin (2015)
\item Mader, Jochen: Vert.x 3: Reactive Microservices, in: Informatik Aktuell (2015)
\item Mader, Jochen: Vert.x 2: das reaktive, modulare Framework, in: Heise Developer (2014)
\item Mader, Jochen: Spring: Modularität als Tugend, in: Java Magazin (2014)
\item Mader, Jochen: Spring Data und QueryDSL, in: Java Magazin (2013)
\item Mader, Jochen: Wicket 6 Bootcamp, in: Java Aktuell (2013)
\item Mader, Jochen: Wicket – Komponentenbasiert und objektorientiert, in: Buch bei Entwickler Press (2012)
\item Mader, Jochen: cURL und REST , in: Java Magazin (2012)
\item Mader, Jochen: Wicket und Activiti , in: Java Magazin (2012)
\item Mader, Jochen: Wicket 6 , in: Java Magazin (2012)
\item Mader, Jochen: Spring, Equinox und RCP , in: Eclipse Magazin (2009)
\item Mader, Jochen: RFID und Java: Funkende Bohnen (Teil 1 + 2) , in: Java Magazin (2010)
\end{enumerate}

\section*{Schulungen}
\begin{longtable}{@{}p{6cm}p{10cm}}
„Spark Introduction“	 & Einführung in die Grundlagen des Spark-Franeworks. \\
„Resilient Architectures“ 	& Diese Schulung wird in Verbindung mit einem Architekturreview durchgeführt.\\
„Reactive Programming with Vert.x“ 	& In aktuellen Kundenprojekten wird diese Schulung verwendet um die beteiligten Entwickler in das notwendige reaktive Denkmuster zu bringen.\\
„Java Concurrency Training“ 	& Basistraining für alle Jene die tiefer in die JVM einsteigen wollen. Kernthemen sind GC, Memory Model, die bekannten und die (leider) weniger bekannten Bausteine aus java.util.concurrent. \\
„Advanced Spring“ 	& Modulare Anwendungsarchitektur basierend auf dem SpringFramework.\\
„Code Quality Management“ 	& Code Quality und Shared Code Ownership sind zwei untrennbar miteinander Verbundene Eigenschaften hochwertiger Software. Im Verlauf dieser Schulung werden sowohl Maßnahmen im Bereich Teamkultur (gemeinsame Codereviews, wechselnde Veratnwortlichkeiten,...) als auch technische Hilfsmittel (Gerrit, SonarQube, ...) vorgestellt und gemeinsam verprobt.\\
\end{longtable}
