\documentclass[
	%pdftex,	% PDFTex verwenden
	a4paper,	% A4 Papier
	%oneside,	% Einseitig
	%bibtotoc,	% Literaturverzeichnis einfügen bibtotocnumbered: nummeriert
	%liststotoc,	% Verzeichnisse einbinden in toc
	%idxtotoc,	% Index ins Verzeichnis einfügen
	%halfparskip,	% Europäischer Satz mit abstand zwischen Absätzen
	%chapterprefix,	% Kapitel anschreiben als Kapitel
	%headsepline,	% Linie nach Kopfzeile
	%footsepline,	% Linie vor Fusszeile
	10pt		% Grössere Schrift, besser lesbar am bildschrim
]{article}

%
% Schrift EurostileLTStd
%
\usepackage{fontspec}
\setmainfont{EurostileLTStd}
\defaultfontfeatures{Ligatures=TeX}

%
% Randabstände einstellen
%
\usepackage[top=4cm, bottom=0.8cm, left=2.25cm, right=2cm]{geometry}

%
% Paket für Übersetzungen ins Deutsche
%
\usepackage[english,ngerman]{babel}

% 
% Datum im Format 19.01.2015
%
\usepackage[ddmmyyyy]{datetime}
\renewcommand{\dateseparator}{.}

%
% Paket um Grafiken einbetten zu können
%
\usepackage{graphicx}

%
% aller Bilder werden im Unterverzeichnis figures gesucht:
%
\graphicspath{{images/}}

%
% Hintergrundbild auf der 1. Seite
%
\usepackage{eso-pic}
\newcommand\BackgroundPic{
 \put(0,0){
  \parbox[b][\paperheight]{\paperwidth}{%
   \vfill
   \centering
   \includegraphics[width=\paperwidth,height=\paperheight,keepaspectratio]{codecentric-title.jpg}%
   \vfill
}}}

%
% Hintergrundbild für die Kopfzeile
%
\usepackage[pages=all]{background}
\backgroundsetup{
placement=top,
scale=0.5,
hshift=320,
vshift=-80,
color=black,
opacity=1.0,
angle=0,
contents={
  \includegraphics[]{codecentric-logo.png}
  }
}

%
% Abschnitte
%
\usepackage{titlesec}
\titleformat{\section}[display]
{\LARGE\color[rgb]{0,0.15,0.34}}
{\thesection.}{0.5em}{}

%
% Tabellenzellen-Abstand = 0
%
\usepackage{array}
\renewcommand{\arraystretch}{2}

%
% Listen mit Abstandsdefinition
%
\usepackage{enumitem}

